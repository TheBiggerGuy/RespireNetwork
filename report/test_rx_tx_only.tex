%%
%% Copyright Guy Taylor 2012
%%
%%
\chapter{\ac{NRF24} Recive and Transmit only Test}

To test and trubleshoot radio communicatiom issues, a radio test mosule was implemented that
contaided code to test simle functions. This module was built utilising the underliying modules
tested to be correct using the digital probe.

\section{Transmit Only}
\lstset{language=[ANSI]C, frame=single, caption={Radio test TX}}
\begin{lstlisting}
struct net_packet_broadcast p;

Radio_init(&local, &broadcast);
Radio_enable(false);
Radio_setMode(Radio_Mode_TX, false);

p.hello[0] = 'h';
p.hello[1] = 'e';
p.hello[2] = 'l';
p.hello[3] = 'l';
p.hello[4] = 'o';

while(true) {
  DBG_probe_on(DBG_Probe_1);
  p.time = RTC_getTime();
  p.tick = RTC_getTickCount();
  Radio_loadbuf_broadcast(&p);
  Radio_enable(true);
  for(volatile uint8_t i=0; i < 9; ++i) {
    __NOP();
  }
  Radio_enable(false);
  while(radio_has_packets_to_sent()){
    __WFI();
  }
  DBG_probe_off(DBG_Probe_1);
  delay(250);
  delay(250); 
}
\end{lstlisting}

\section{Recive Only}

\lstset{language=[ANSI]C, frame=single, caption={Radio test RX}}
\begin{lstlisting}
Radio_init(&local, &broadcast);
Radio_enable(false);
Radio_setMode(Radio_Mode_RX, false);
Radio_enable(true);
while(true){
  if(Radio_available() > 0) {
    struct net_packet_broadcast buf;
    Radio_recive((uint8_t *) &buf, sizeof(struct net_packet_broadcast));
    DBG_probe_toggle(DBG_Probe_1);
  }
}
\end{lstlisting}


