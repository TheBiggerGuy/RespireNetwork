%%
%% Copyright Guy Taylor 2012
%%
%%

\chapter{Conclusion}

Although this project did not produce a final complete networking solution for the Respire, it has
developed firmware and analysed important aspects of system design that future work can build
upon. In particular, it has identified the onboard \ac{NRF24} radio as a weak link in the system required
to effect this particular solution. The Respire network is an excellent concept for improving patient
management and care (and also conceivably as a future as a health/fitness aid) and \ac{TDMA} is a
preferred protocol for implementing this network. If one of the new generation of low-power radios
(\eg Bluetooth Low Energy) is a suitable replacement to the \ac{NRF24} as a Respire device
modification, then I am confident that the firmware, debugging protocols and hardware connections
I have developed during this project will greatly facilitate testing and implementation of these
devices in a Respire network.

